\documentclass[12pt]{article}
\usepackage[spanish]{babel}
\usepackage[utf8]{inputenc}
\usepackage[T1]{fontenc}
\usepackage{geometry}
\usepackage{xcolor}
\usepackage{listings}
\usepackage{graphicx}
\usepackage{titlesec}
\usepackage{fancyhdr}

\geometry{a4paper, margin=2.5cm}
\definecolor{codegreen}{rgb}{0,0.6,0}
\definecolor{codegray}{rgb}{0.5,0.5,0.5}
\definecolor{codepurple}{rgb}{0.58,0,0.82}
\definecolor{backcolour}{rgb}{0.95,0.95,0.92}

\lstdefinestyle{mystyle}{
	backgroundcolor=\color{backcolour},
	commentstyle=\color{codegreen},
	keywordstyle=\color{magenta},
	numberstyle=\tiny\color{codegray},
	stringstyle=\color{codepurple},
	basicstyle=\ttfamily\footnotesize,
	breakatwhitespace=false,         
	breaklines=true,                 
	captionpos=b,                    
	keepspaces=true,                 
	numbers=left,                    
	numbersep=5pt,                  
	showspaces=false,                
	showstringspaces=false,
	showtabs=false,                  
	tabsize=2,
	frame=single,
	frameround=tttt
}

\lstset{style=mystyle}

% Formato de secciones
\titleformat{\section}{\large\bfseries\color{blue}}{\thesection}{1em}{}
\titleformat{\subsection}{\normalsize\bfseries\color{teal}}{\thesubsection}{1em}{}

% Encabezado y pie de página
\pagestyle{fancy}
\fancyhf{}
\rhead{Práctica 03 - Sistemas de Gestión de Bases de Datos}
\lhead{Angello Marcelo Zamora Valencia}
\rfoot{\thepage}

\begin{document}
	
	\begin{titlepage}
		\centering
		\vspace*{1cm}
		
		\includegraphics[width=0.4\textwidth]{logo_unap.png}\par
		\vspace{1cm}
		
		{\scshape\LARGE Universidad Nacional del Altiplano \par}
		\vspace{0.5cm}
		
		{\scshape\Large Escuela Profesional de Ingeniería Estadística e Informática\par}
		\vspace{1.5cm}
		
		{\huge\bfseries Práctica 03: Consultas SQL\par}
		\vspace{2cm}
		
		{\Large\itshape Angello Marcelo Zamora Valencia\par}
		\vfill
		\vspace{0.5cm}
		
		{\large Vto ciclo\par}
		{\large Sistemas de Gestión de Bases de Datos\par}
		{\large \today\par}
	\end{titlepage}
	
	\section{Introducción}
	En el presente documento se detallan las consultas SQL realizadas como parte de la Práctica 03 del curso de Sistemas de Gestión de Bases de Datos. Se trabajó con dos bases de datos principales: \texttt{bd\_restaurant} y \texttt{bdmuniate\_multas}, realizando diversas consultas que permiten extraer información valiosa para la gestión de estos sistemas.
	
	\section{Consultas en BD\_Restaurant}
	
	\subsection{Especialidades del restaurante}
	Consulta para obtener las especialidades de los empleados:
	\begin{lstlisting}[language=SQL]
		SELECT especialidad FROM empleados;
	\end{lstlisting}
	
	\subsection{Tipo de bebida y comida que expende el restaurante}
	Consulta para conocer los tipos de productos vendidos:
	\begin{lstlisting}[language=SQL]
		SELECT tipo FROM ventas;
	\end{lstlisting}
	
	\subsection{Ingreso total según tipo de venta}
	Consulta para calcular los ingresos totales:
	\begin{lstlisting}[language=SQL]
		SELECT SUM(costo) AS total_costos
		FROM ventas;
	\end{lstlisting}
	
	\subsection{Empleado que recibió más propinas}
	Consulta para identificar al empleado con mayores propinas:
	\begin{lstlisting}[language=SQL]
		SELECT 
		id_empleado,
		SUM(propina) AS total_propina
		FROM 
		registro_caja
		GROUP BY 
		id_empleado
		ORDER BY 
		total_propina DESC
		LIMIT 1;
	\end{lstlisting}
	
	\section{Consultas en BD\_MuniaMultas}
	
	\subsection{Ordenamiento alfabético de registros}
	Consultas para ordenar los datos de multas:
	
	\begin{lstlisting}[language=SQL]
		-- Por departamento y provincia
		SELECT DISTINCT departamento, provincia
		FROM bdmuniate_multas
		ORDER BY departamento ASC, provincia ASC;
		
		-- Por zona
		SELECT DISTINCT zona
		FROM bdmuniate_multas
		ORDER BY zona ASC;
		
		-- Por tipo de establecimiento (giro)
		SELECT DISTINCT giro AS tipo_establecimiento
		FROM bdmuniate_multas
		ORDER BY giro ASC;
		
		-- Consulta combinada
		SELECT departamento, provincia, zona, giro AS tipo_establecimiento
		FROM bdmuniate_multas
		ORDER BY departamento ASC, provincia ASC, zona ASC, giro ASC;
	\end{lstlisting}
	
	\subsection{Monto total recaudado por departamento}
	Consulta para calcular el monto total de multas por departamento:
	\begin{lstlisting}[language=SQL]
		SELECT 
		departamento,
		SUM(total) AS monto_total_recaudado
		FROM 
		bdmuniate_multas
		GROUP BY 
		departamento
		ORDER BY 
		monto_total_recaudado DESC;
	\end{lstlisting}
	
\end{document}